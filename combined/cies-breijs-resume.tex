% LaTeX source of my resume
% =========================

% Commented for easy reuse... ;)

% See the `README.md` file for more info.

% This file is licensed under the CC-NC-ND Creative Commons license.


% start a document with the here given default font size and paper size
\documentclass[10pt,a4paper]{article}

% include the `tex` instructions that takes care of loading packages and defining commands

% Copyright (c) 2012 Cies Breijs
%
% The MIT License
%
% Permission is hereby granted, free of charge, to any person obtaining a copy
% of this software and associated documentation files (the "Software"), to deal
% in the Software without restriction, including without limitation the rights
% to use, copy, modify, merge, publish, distribute, sublicense, and/or sell
% copies of the Software, and to permit persons to whom the Software is
% furnished to do so, subject to the following conditions:
%
% The above copyright notice and this permission notice shall be included in
% all copies or substantial portions of the Software.
%
% THE SOFTWARE IS PROVIDED "AS IS", WITHOUT WARRANTY OF ANY KIND, EXPRESS OR
% IMPLIED, INCLUDING BUT NOT LIMITED TO THE WARRANTIES OF MERCHANTABILITY,
% FITNESS FOR A PARTICULAR PURPOSE AND NONINFRINGEMENT. IN NO EVENT SHALL THE
% AUTHORS OR COPYRIGHT HOLDERS BE LIABLE FOR ANY CLAIM, DAMAGES OR OTHER
% LIABILITY, WHETHER IN AN ACTION OF CONTRACT, TORT OR OTHERWISE, ARISING FROM,
% OUT OF OR IN CONNECTION WITH THE SOFTWARE OR THE USE OR OTHER DEALINGS IN THE
% SOFTWARE.


% Some commands for making a LaTeX resume
% =======================================

% Commented ;)

% See the README.md file for more info



% \documentclass[10pt,a4paper]{article}  % i do this in the document itself


%%% LOAD AND SETUP PACKAGES

\usepackage[a4paper,margin=0.75in]{geometry}
\usepackage{mdwlist}   % to finetue lists with a inline heading and indented content (see Experiences)
\usepackage{multicol}  % for multiple column text
\usepackage{relsize}   % for \textscale, which I prefer over \sc (small caps), see my \acr command
\usepackage[english]{babel}
\hyphenation{Some-long-word}

\usepackage{hyperref}
%\usepackage[pdftex]{hyperref} % yups, URLs everwhere...
\usepackage{xcolor}  % ... and color them links
\definecolor{dark-blue}{rgb}{0.15,0.15,0.4}
\hypersetup{colorlinks,linkcolor={dark-blue},citecolor={dark-blue},urlcolor={dark-blue}}

\usepackage{ifxetex}
\ifxetex
  \usepackage{fontspec}
  \setmainfont
    [ ExternalLocation = /Users/sebinsua/Library/Fonts/,
      Mapping          = tex-text ,
      Numbers          = OldStyle ,
      Ligatures        = {Common,Contextual} ,
      UprightFont      = *-regular ,
      BoldFont         = *-bold ,
      ItalicFont       = *-italic ,
      BoldItalicFont   = *-bolditalic ]
    {texgyrepagella}
  % Comment out the previous statement and uncomment the following line to use the
  % Linux Libertine font (it has nice lignatures).
  % Make sure to have the `ttf-linux-libertine` package installed on Ubuntu.
  \setmainfont[Mapping=tex-text, Numbers=OldStyle, Ligatures={Common,Contextual}]{Linux Libertine O}
  \usepackage[protrusion]{microtype}  % needs an experimental and impposible to find package for xetex
\else
  \usepackage{tgpagella}  % this case we lack lower case numbers, ligatures and some typographic niceties
  \usepackage[expansion,protrusion]{microtype}
\fi



%%% DOCUMENT WIDE STYLING

\pagestyle{empty}
\setlength{\tabcolsep}{0em}
\xspaceskip7pt  % some more spacing between sentences (use "i.e.\ " or "with SQL\@. " in case of errors)


%%% CUSTOM COMMANDS

% main title (name) with subtitle (date)
\newcommand*\maintitle[2]{\noindent{\LARGE \textbf{#1}}\ \ \ \emph{#2}}

% title for the root sections (experience, education, etc) of the resume
\newcommand*\roottitle[1]{\subsection*{#1}\vspace{-0.3em}\nopagebreak[4]}

% acr command, to quickly mark acronyms for special formatting
\newcommand*\acr[1]{\textscale{.85}{#1}}

% pretty bullet (created from a much smaller centerdot), \sbull contains its spacing
\newcommand*\bull{\raisebox{-0.365em}[-1em][-1em]{\textscale{4}{$\cdot$}}}
\newcommand*\sbull{\ \ \bull \ \ }

% it seems not to work when simply using \parindent...
\newlength{\newparindent}
\addtolength{\newparindent}{\parindent}

% a double \parindent...
\newlength{\doubleparindent}
\addtolength{\doubleparindent}{\parindent}
\addtolength{\doubleparindent}{\parindent}

% indentsection style, used for sections that aren't already in lists
% that need indentation to the level of all text in the document
\newenvironment{indentsection}%
{\begin{list}{}%
  {\setlength{\leftmargin}{\newparindent}\setlength{\parsep}{0pt}\setlength{\parskip}{0pt}\setlength{\itemsep}{0pt}\setlength{\topsep}{0pt}}%
}
{\end{list}}

% headerrow command, used for a new employer
\newcommand{\headedsection}[3]{\nopagebreak[4]\begin{indentsection}\item[]\textscale{1.1}{#1}\hfill#2#3\end{indentsection}\nopagebreak[4]}

% subheaderrow command, used for a new position
\newcommand{\headedsubsection}[3]{\nopagebreak[4]\begin{indentsection}\item[]\textbf{#1}\hfill\emph{#2}#3\end{indentsection}\nopagebreak[4]}

% body text (indented)
\newcommand{\bodytext}[1]{\nopagebreak[4]\begin{indentsection}\item[]#1\end{indentsection}\pagebreak[2]}

% \vspace variaties
\newcommand{\breakvspace}[1]{\pagebreak[2]\vspace{#1}\pagebreak[2]}
\newcommand{\nobreakvspace}[1]{\nopagebreak[4]\vspace{#1}\nopagebreak[4]}

% \spacedhrule a horizontal line with some vertical space before and after it
\newcommand{\spacedhrule}[2]{\breakvspace{#1}\hrule\nobreakvspace{#2}}

% \inlineheadsection command, used for a new employer
\newcommand{\inlineheadsection}[2]{\begin{basedescript}{\setlength{\leftmargin}{\doubleparindent}}\item[\hspace{\newparindent}\textbf{#1}]#2\end{basedescript}\vspace{-1.7em}}

% apo command, for an apostrophe that looks good on old style nums
\newcommand{\apo}{\raisebox{-.18ex}{'}{\hspace{-.1em}}}

% non space that allows line breaks
\newcommand*{\nsp}{\hskip0pt}

%%% MORE SPECIFIC COMMANDS

% CPP command (found it in some corner of the internet and decided to use it)
\newcommand{\CPP}{C\nolinebreak[4]\hspace{-.04em}\raisebox{.20ex}{\footnotesize\bf++} }

% KTurtle command :)
\newcommand{\KTurtle}{\acr{KT}urtle }



% % these are in the document itself:
%
% \begin{document}
% ...the document text...
% \end{document}



\begin{document}  % begin the content of the document
\sloppy  % this to relax whitespacing in favour of straight margins

\maintitle{Cies Breijs}{June 12, 1982}  % title on top of the document

\nobreakvspace{0.3em}  % add some page break averse vertical spacing

% \noindent prevents paragraph's first lines from indenting
% \mbox is used to obfuscate the email address
% \sbull is a spaced bullet
% \href well..
% \\ breaks the line into a new paragraph
\noindent\href{mailto:cies.at.kde.dot.nl}{cies\mbox{}@\mbox{}kde.nl}\sbull
\textsmaller{+}31.646469087\sbull
cies010 \emph{(Skype)}\sbull
\href{http://www.linkedin.com/in/ciesbreijs}{www.linkedin.com/in/ciesbreijs}
\\
Mathenesserplein 84\sbull
3022\thinspace {\sc ld}\sbull
Rotterdam\sbull
The Netherlands

\spacedhrule{0.9em}{-0.4em}  % a horizontal line with some vertical spacing before and after

\roottitle{Summary}  % a root section title

\vspace{-1.3em}  % some vertical spacing
\begin{multicols}{2}  % open a multicolumn environment
\noindent \emph{Creative geek with roots in the open source movement, an entrepreneurial mindset and a passion for user/customer centered software development with maintainable outcomes.}
\\
\\
At the age of seven (1989) I wrote my first lines of code in a \acr{LOGO}-like language on an \acr{MSX} (pre-\acr{PC}).  Two years later I attended a conference on an emerging new technology, the Internet, at the Erasmus University from which I would graduate 16 years later.

After being introduced to the open source movement in 1997, I taught myself a variety of skills with help from the open source community.  Since 2002 I am a contributor to the \acr{KDE} project, as my pet project \KTurtle got admitted into their \emph{edu} module.

\KTurtle is a zero-entry-barrier programming environment that is shipped with every Linux distribution.  It is well suited for teaching (children) the basics of programming, math and geometry.  Through this project I hope to give others the same opportunity I was given:\ to learn programming at early age.

From 2003 to 2007 I studied at the Erasmus University Rotterdam and graduated in \emph{Business and Computer Science} (one curriculum).  After graduation I traveled Europe and Asia during a two year sabbatical, on which I worked on several occasions (see experience below).  Most challenging was my work for Intellecap:\ I architected and implemented an open source administrative backend for the booming micro credit lending industry.

Upon return from my travels (\apo09) I joined Zarafa, one of the largest open source software vendors in the Netherlands.  In 2011 was asked to join Intellecap as the \acr{CTO} of their software division (\acr{ISTPL}) that within half a year had to let me go due to restructuring.  Currently I am looking for a challenging position to further train my geek muscle.
\end{multicols}


\spacedhrule{0em}{-0.4em}

\roottitle{Experience}

\headedsection  % sets the header for the section and includes any subsections
  {\href{http://www.intellecap.com}{Intellecap}/\href{http://istpl.in}{\acr{ISTPL}}}
  {\textsc{Mumbai, Pune \& Hyderabad, India}} {%

  \headedsubsection  % sets the header for a subsection and contains usually body text
    {\acr{IT} Consultant}
    {Nov \apo08 -- Feb \apo09}
    {\bodytext{Intellecap is a social-sector advisory firm serving corporates, non-profits, development agencies and governments working in developing markets. I assessed their software development team and methodologies, trained their developers and build several web applications. One of those apps is \href{http://www.mostfit.org}{Mostfit}, an open source \acr{MIS} for \href{http://en.wikipedia.org/wiki/Microcredit}{microcredit} lenders.}}

  \headedsubsection
    {\acr{IT} \& Strategy Consultant}
    {Jan \apo10 -- Aug \apo11}
    {\bodytext{\href{http://www.mostfit.org}{Mostfit} became a success and I was called in to solve several technical challenges, look at potential growth strategies, and assist in making a business case for this product in a company of its own.}}

  \headedsubsection
    {Chief Technology Officer}
    {Oct \apo11 -- present}
    {\bodytext{Joined the \acr{C}-family to make \href{http://www.mostfit.org}{Mostfit} the nr.1 software solution for micro credit lenders around the globe.}}
}

\headedsection
  {\href{http://www.zarafa.com}{Zarafa}}
  {\textsc{Delft, The Netherlands}} {%
  \headedsubsection
    {\acr{QA} \& Release Manager}
    {Dec \apo09 -- Jan \apo11}
    {\bodytext{Zarafa might be the fastest growing open source product company in Europe. For an open source enthusiast like myself, working with Zarafa is a wonderful opportunity. Reporting directly to the \acr{CEO}, Brian Josef, and working closely with the \acr{CTO}, Steve Hardy, I drove the change to test automation and continuous integration. Furthermore I architected new documentation and translation systems. My Indian work experiences proved particularly valuable, as I was sent to analyze and streamline their outsourced operations in India.}}
}

\headedsection
  {\href{http://www.dharmapublishing.com}{Dharma Publishing}}
  {\textsc{near San Francisco (\acr{CA}), \acr{USA}}} {%
  \headedsubsection
    {\acr{IT} Consultant}
    {Nov \apo09 -- Dec \apo09}
    {\bodytext{Dharma Publishing, the worlds largest Buddhist publisher, is a non-profit, all-volunteer organization that helps to preserve Tibetan Buddhism and culture. I build them a \href{http://www.dharmapublishing.com}{web store}, and moved the infrastructure of all their web media to a nearly zero maintenance setup that they can maintain themselves.}}
}

\headedsection
  {\href{http://www.kde.org}{KDE}}
  {\href{http://edu.kde.org/kturtle}{edu.kde.org/kturtle}} {%
  \headedsubsection
    {Software Engineer}
    {Dec \apo03 -- present}
    {\bodytext{\KTurtle is an educational programming environment that simplifies learning the basics of programming. At early age I learned programming using \acr{MSX-LOGO}, by creating \KTurtle I want to ensure future generation's access to a similar piece of software. I was honored when \KTurtle got admitted to \acr{KDE} in 2003.}}
}

\headedsection
  {\href{http://truetopiaproject.org}{Truetopia Project}}
  {\href{http://truetopiaproject.org}{truetopiaproject.org}} {%
  \headedsubsection
    {Initiator}
    {Nov \apo07 -- Apr \apo10}
    {\bodytext{The Truetopia Project is an open source web application (Rails) to facilitate self-governing communities. It provides a workflow for collaborative problem identification and solution design.}}
}

\headedsection
  {\href{http://www.dpu.ac.th/dpuic}{Dhurakij Pundit University International College}}
  {\textsc{Bangkok, Thailand}} {%
  \headedsubsection
    {Guest Lecturer}
    {Sep \apo09}
    {\bodytext{Dr.\@ Pilun Piyasirivej and Mr.\@ Michel Bauwens invited me to give two guest lectures:\ one on the open source movement and one on the semantic web.}}
}

\headedsection
  {\href{http://www.opendream.th}{Opendream}}
  {\textsc{Bangkok, Thailand}} {%
  \headedsubsection
    {\acr{IT} Consultant}
    {Aug \apo09 -- Sep \apo09}
    {\bodytext{Architected and largely implemented an open source media sharing web service (\acr{REST} api) that facilitates video uploads, transcoding and streaming. Coached their development team on system design, Ruby development (using Merb/Rails) and testing strategies such as \acr{TDD}/\acr{BDD}.}}
}

\headedsection
  {\href{http://www.commuun.nl}{Commuun}}
  {\textsc{Rotterdam, The Netherlands}} {%
  \headedsubsection
    {Senior Visionary}
    {Jul \apo06 -- Sep \apo09}
    {\bodytext{I helped Peter Duijnstee (the proprietor of Commuun) with setting up the technical infrastructure, defining the core competences and creating a brand for his internet company. We collaborated on several web applications (all Rails apps) within the context of his company.}}
}

\headedsection
  {\href{http://www.eur.nl}{Erasmus University Rotterdam}}
  {\textsc{Rotterdam, The Netherlands}} {%
  \headedsubsection
    {Guest Lecturer}
    {Jul \apo06 -- Jul \apo09}
    {\bodytext{Yearly guest lectures on the phenomenon of open source, as part of the first year curriculum of \emph{Computer Science \& Economics}.}}
}

\headedsection
  {LIP Automatisering}
  {\textsc{Breda, The Netherlands}} {%
  \headedsubsection
    {Software Auditor}
    {Sep \apo06}
    {\bodytext
      {Audited their flag ship product \emph{\acr{LIP} Suite}:\ an \acr{ERP} solution for construction companies.}}
}

\headedsection
  {\href{http://www.thehealthagency.com}{The Health Agency}}
  {\textsc{Delft \& Rotterdam, The Netherlands}} {%

  \headedsubsection
    {Software Engineer}
    {Jun \apo05 -- Feb \apo06}
    {\bodytext{I worked on their innovative content management system (\acr{CMS}) targeted towards the health care sector. It is developed from scratch in Python and uses Postgre\acr{SQL}, \acr{XML}/\acr{XSLT} and Twisted.}}

  \headedsubsection
    {Software Auditor}
    {Dec \apo06}
    {\bodytext{Assessed their Python/Zope/\acr{Z}o\acr{DB}-based web framework re-engineering project.}}
}

\begin{center}
  \emph{Please refer to \href{http://www.linkedin.com/in/ciesbreijs}{my Linkedin profile} for the complete list of work experiences along with recommendations.}
\end{center}


\spacedhrule{-0.2em}{-0.4em}

\roottitle{Education}

\headedsection
  {Erasmus University Rotterdam}
  {\textsc{Rotterdam, The Netherlands}} {%
  \headedsubsection
    {Bachelor degree in Computer Science \& Economics}
    {2004 -- 2007}
    {\bodytext{Focussed on the economics of open source, rapid application development (\acr{RAD}) and the semantic web technology stack (\acr{RDF}/\acr{RDFS}, \acr{OWL} and \acr{SPARQL}).}}
}

\headedsection
  {Technical University Delft}
  {\textsc{Delft, The Netherlands}} {%
  \headedsubsection
    {Industrial Design Engineering \textnormal{(discontinued)}}
    {2001 -- 2002} {}
}

\headedsection
  {Libanon Lyceum}
  {\textsc{Rotterdam, The Netherlands}} {%
  \headedsubsection
    {\acr{VWO} \textnormal{(pre-university secondary education)}}
    {1994 -- 2000} {}
}


\spacedhrule{0.5em}{-0.4em}

\roottitle{Skills}

\inlineheadsection  % special section that has an inline header with a 'hanging' paragraph
  {Technical specialties:}
  {Software design and implementation, with(in) a team. I love Ruby/\nsp Python/\nsp Java/\nsp \CPP and flirt regularly with Haskell. Solid knowledge of web technologies:\ \acr{HTML+CSS}, \acr{XML}, \acr{RDF}, \acr{REST}, \acr{SOAP} and JavaScript (mainly jQuery). Linux administration skills:\ bash, Apache, My\acr{SQL}, Postgres\acr{SQL}, virtualization/cloud (Open\acr{VZ}, \acr{VM}ware, \acr{KVM}, Xen and \acr{EC}2), datacenter automation (Puppet and Chef), continuous integration (Hudson/Jenkins).}

\inlineheadsection
  {Natural languages:}
  {Dutch \emph{(mother tongue)}, English \emph{(full professional proficiency)}, German \emph{(limited working proficiency)}, French \emph{(elementary proficiency)} and Mandarin Chinese \emph{(beginner)}.}


\spacedhrule{1.6em}{-0.4em}

\roottitle{Interests}

\inlineheadsection
  {Non-exhaustive and in alphabetical order:}
  {Art, Buddhism, cryptography, music, open source, philosophy, software engineering, travel, typography (e.g.\ graphic design, \LaTeX), \acr{UI}/\acr{UX} and vegetarian cooking.}


\end{document}
