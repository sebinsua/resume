% LaTeX source of my resume
% =========================

% Commented for easy reuse... ;)

% See the `README.md` file for more info.

% This file is licensed under the CC-NC-ND Creative Commons license.


% start a document with the here given default font size and paper size
\documentclass[10pt,a4paper]{article}

% include the `tex` instructions that takes care of loading packages and defining commands
% Copyright (c) 2012 Cies Breijs
%
% The MIT License
%
% Permission is hereby granted, free of charge, to any person obtaining a copy
% of this software and associated documentation files (the "Software"), to deal
% in the Software without restriction, including without limitation the rights
% to use, copy, modify, merge, publish, distribute, sublicense, and/or sell
% copies of the Software, and to permit persons to whom the Software is
% furnished to do so, subject to the following conditions:
%
% The above copyright notice and this permission notice shall be included in
% all copies or substantial portions of the Software.
%
% THE SOFTWARE IS PROVIDED "AS IS", WITHOUT WARRANTY OF ANY KIND, EXPRESS OR
% IMPLIED, INCLUDING BUT NOT LIMITED TO THE WARRANTIES OF MERCHANTABILITY,
% FITNESS FOR A PARTICULAR PURPOSE AND NONINFRINGEMENT. IN NO EVENT SHALL THE
% AUTHORS OR COPYRIGHT HOLDERS BE LIABLE FOR ANY CLAIM, DAMAGES OR OTHER
% LIABILITY, WHETHER IN AN ACTION OF CONTRACT, TORT OR OTHERWISE, ARISING FROM,
% OUT OF OR IN CONNECTION WITH THE SOFTWARE OR THE USE OR OTHER DEALINGS IN THE
% SOFTWARE.


% Some commands for making a LaTeX resume
% =======================================

% Commented ;)

% See the README.md file for more info



% \documentclass[10pt,a4paper]{article}  % i do this in the document itself


%%% LOAD AND SETUP PACKAGES

\usepackage[a4paper,margin=0.75in]{geometry}
\usepackage{mdwlist}   % to finetue lists with a inline heading and indented content (see Experiences)
\usepackage{multicol}  % for multiple column text
\usepackage{relsize}   % for \textscale, which I prefer over \sc (small caps), see my \acr command
\usepackage[english]{babel}
\hyphenation{Some-long-word}

\usepackage{enumitem}

\usepackage{hyperref}
%\usepackage[pdftex]{hyperref} % yups, URLs everwhere...
\usepackage{xcolor}  % ... and color them links
\definecolor{dark-blue}{rgb}{0.15,0.15,0.4}
\hypersetup{colorlinks,linkcolor={dark-blue},citecolor={dark-blue},urlcolor={dark-blue}}

\usepackage{ifxetex}
\ifxetex
  \usepackage{fontspec}
  \setmainfont
    [ ExternalLocation = /Users/sebinsua/Library/Fonts/,
      Mapping          = tex-text ,
      Numbers          = OldStyle ,
      Ligatures        = {Common,Contextual} ,
      UprightFont      = *-regular ,
      BoldFont         = *-bold ,
      ItalicFont       = *-italic ,
      BoldItalicFont   = *-bolditalic ]
    {texgyrepagella}
  % Comment out the previous statement and uncomment the following line to use the
  % Linux Libertine font (it has nice lignatures).
  % Make sure to have the `ttf-linux-libertine` package installed on Ubuntu.
  \setmainfont[Mapping=tex-text, Numbers=OldStyle, Ligatures={Common,Contextual}]{Linux Libertine O}
  \usepackage[protrusion]{microtype}  % needs an experimental and impposible to find package for xetex
\else
  \usepackage{tgpagella}  % this case we lack lower case numbers, ligatures and some typographic niceties
  \usepackage[expansion,protrusion]{microtype}
\fi



%%% DOCUMENT WIDE STYLING

\pagestyle{empty}
\setlength{\tabcolsep}{0em}
\xspaceskip7pt  % some more spacing between sentences (use "i.e.\ " or "with SQL\@. " in case of errors)


%%% CUSTOM COMMANDS

% main title (name) with subtitle (date)
\newcommand*\maintitle[2]{\noindent{\LARGE \textbf{#1}}\ \ \ \emph{#2}}

% title for the root sections (experience, education, etc) of the resume
\newcommand*\roottitle[1]{\subsection*{#1}\vspace{-0.3em}\nopagebreak[4]}

% acr command, to quickly mark acronyms for special formatting
\newcommand*\acr[1]{\textscale{.85}{#1}}

% pretty bullet (created from a much smaller centerdot), \sbull contains its spacing
\newcommand*\bull{\raisebox{-0.365em}[-1em][-1em]{\textscale{4}{$\cdot$}}}
\newcommand*\sbull{\ \ \bull \ \ }

% it seems not to work when simply using \parindent...
\newlength{\newparindent}
\addtolength{\newparindent}{\parindent}

% a double \parindent...
\newlength{\doubleparindent}
\addtolength{\doubleparindent}{\parindent}
\addtolength{\doubleparindent}{\parindent}

% indentsection style, used for sections that aren't already in lists
% that need indentation to the level of all text in the document
\newenvironment{indentsection}%
{\begin{list}{}%
  {\setlength{\leftmargin}{\newparindent}\setlength{\parsep}{0pt}\setlength{\parskip}{0pt}\setlength{\itemsep}{0pt}\setlength{\topsep}{0pt}}%
}
{\end{list}}

% headerrow command, used for a new employer
\newcommand{\headedsection}[3]{\nopagebreak[4]\begin{indentsection}\item[]\textscale{1.1}{#1}\hfill#2#3\end{indentsection}\nopagebreak[4]}

% subheaderrow command, used for a new position
\newcommand{\headedsubsection}[3]{\nopagebreak[4]\begin{indentsection}\item[]\textbf{#1}\hfill\emph{#2}#3\end{indentsection}\nopagebreak[4]}

% body text (indented)
\newcommand{\bodytext}[1]{\nopagebreak[4]\begin{indentsection}\item[]#1\end{indentsection}\pagebreak[2]}

% \vspace variaties
\newcommand{\breakvspace}[1]{\pagebreak[2]\vspace{#1}\pagebreak[2]}
\newcommand{\nobreakvspace}[1]{\nopagebreak[4]\vspace{#1}\nopagebreak[4]}

% \spacedhrule a horizontal line with some vertical space before and after it
\newcommand{\spacedhrule}[2]{\breakvspace{#1}\hrule\nobreakvspace{#2}}

% \inlineheadsection command, used for a new employer
\newcommand{\inlineheadsection}[2]{\begin{basedescript}{\setlength{\leftmargin}{\doubleparindent}}\item[\hspace{\newparindent}\textbf{#1}]#2\end{basedescript}\vspace{-1.7em}}

% apo command, for an apostrophe that looks good on old style nums
\newcommand{\apo}{\raisebox{-.18ex}{'}{\hspace{-.1em}}}

% non space that allows line breaks
\newcommand*{\nsp}{\hskip0pt}

%%% MORE SPECIFIC COMMANDS

% CPP command (found it in some corner of the internet and decided to use it)
\newcommand{\CPP}{C\nolinebreak[4]\hspace{-.04em}\raisebox{.20ex}{\footnotesize\bf++} }

% KTurtle command :)
\newcommand{\KTurtle}{\acr{KT}urtle }



% % these are in the document itself:
%
% \begin{document}
% ...the document text...
% \end{document}




\begin{document}  % begin the content of the document
\sloppy  % this to relax whitespacing in favour of straight margins

\maintitle{Seb Insua}{JavaScript Engineer}  % title on top of the document

\nobreakvspace{0.3em}  % add some page break averse vertical spacing

% \noindent prevents paragraph's first lines from indenting
% \mbox is used to obfuscate the email address
% \sbull is a spaced bullet
% \href well..
% \\ breaks the line into a new paragraph
\noindent\textsmaller{+}44 79 123 03758\sbull
\href{http://github.com/sebinsua}{github.com/sebinsua}\sbull
\href{http://twitter.com/sebinsua}{twitter.com/sebinsua}\sbull
\href{http://sebinsua.com}{sebinsua.com}\sbull
\href{mailto:me@sebinsua.com}{me\mbox{}@\mbox{}sebinsua.com}
\\
6 De Beauvoir Rd\sbull
Hackney\sbull
London\sbull
N1 5SU

\spacedhrule{0.9em}{-0.4em}  % a horizontal line with some vertical spacing before and after

\roottitle{Summary}  % a root section title

\vspace{-1.3em}  % some vertical spacing
\begin{multicols}{2}  % open a multicolumn environment
\noindent \emph{Software engineer and systems thinker with a deep interest in the humanities, a passion for product design and an entrepreneurial mindset.}
\\
\\
I was sixteen when I wrote my first few lines of code. My interest had been piqued when I discovered a small homebrew community for the handheld gaming device, the GP32, and it was there my journey as an engineer started. Nowadays I prefer developing web applications but I still maintain that we learn best through play.\newline

As a self-directed learner I try to avoid becoming too trapped in one domain and direct my learning towards fundamentals and principles that are cross-domain. This occasionally has lead me in strange directions, such as the time I bootstrapped a non-profit online music magazine with 15 contributors; but often I find that experiences in one domain give insight in others. With software engineering I favour the philosophy and techniques from: \href{http://12factor.net}{the twelve-factor app}, functional programming, antifragility, behaviour-driven development (\acr{BDD}) and \href{http://infoq.com/presentations/Simple-Made-Easy}{Simple Made Easy}.\newline

I've worked with a lot of different technologies both back-end and front-end but nowadays I am focusing on Node.JS, JavaScript and Clojure - the former two in which I consider myself an expert and the latter a passionate beginner.\newline

Last year I failed in creating a startup but ended up learning how to run customer development, how to pitch a concept to potential employees/investors, how to create an iOS app with Objective-C and business/marketing stuff like \acr{CAC}, \acr{ARPU}, \acr{LTV}.

\end{multicols}

\inlineheadsection  % special section that has an inline header with a 'hanging' paragraph
  {Technical specialties:}
  {Full stack software design and implementation. I love to work on on the back-end creating \acr{REST}ful \acr{API}s using Node.JS or Python; but I also know front-end technologies such as JavaScript, Angular.JS, Backbone.JS, \acr{HTML5} and \acr{CSS3}.}

\inlineheadsection  % special section that has an inline header with a 'hanging' paragraph
  {Other experience:}
  {I love coding JavaScript in a functional programming style and have also been picking up a little Clojure in my spare time. I follow the mantra \emph{the right tool for the right job} and quickly pick things up. In the past I've worked with: angular.js, Bower, Leaflets.js, parsley.js, grunt.js, sass, stylus, require.js, jade, backbone.js, Objective-C, \acr{TDD} \& \acr{BDD} (PHPUnit, unittest, Jasmine, mocha.js, qunit), express.js, underscore.js, Q.js, async.js, \LaTeX, Flask, Django, My\acr{SQL}, Postgre\acr{SQL}, Python, Java, Cassandra, MongoDB, etc.}

\inlineheadsection  % special section that has an inline header with a 'hanging' paragraph
  {Leadership, operations \& administration:}
  {Experience in team leadership (lead teams of 5+ people), shell scripting, Vagrant, Apache, Nginx, datacenter automation (Puppet), and continuous integration (Jenkins, Travis).}

\spacedhrule{1.5em}{-0.4em}

\roottitle{Experience}

\headedsection  % sets the header for the section and includes any subsections
  {\href{http://www.momentumlabs.io}{Momentum Labs}}
  {\textsc{London, United Kingdom}} {%

  \headedsubsection  % sets the header for a subsection and contains usually body text
    {Director \& Chief Hacker}
    {Jun \apo13 -- Present}
    {\bodytext{\emph{Formed during my sabbatical for my contracting/freelancing activities as well as to act as a hub for startup ideas and open-source projects.} 
    \begin{itemize}[leftmargin=0cm]
      \item My first project was a \href{http://getawayapp.co}{service that allows you to take spontaneous weekend breaks}. After the initial customer development and market research stage, I formed a business model and with the help of a mobile app designer began coding a native app for iOS (the app is unavailable on the App Store as the project has been on-hold for the past few months and e-commerce elements need to be removed before it can be released).
      \item Other projects include \href{https://github.com/sebinsua/express-keenio}{an Express middleware to allow one-line installation of keen.io analytics into Node.JS projects}, a JavaScript object transformation library called \href{https://github.com/sebinsua/jstruct}{Jstruct}, and \href{http://thingsiwouldneverdo.com}{a prototype angular.js app for charity}.
    \end{itemize}
  }}
}

\headedsection
  {\href{http://www.hailocab.com}{Hailo}}
  {\textsc{London, United Kingdom}} {%
  \headedsubsection
    {Contract JavaScript Engineer}
    {Sep \apo13 -- Jan \apo14}
    {\bodytext{\emph{Hailo is the evolution of the hail -- a free smartphone app which puts people just two taps away from a licensed taxi, and lets cabbies get more passengers when they want them.}
    \begin{itemize}[leftmargin=0cm]
      \item Work on a series of single-page dashboard apps. Due to an extensive service orientated architecture a lot of HTTP requests were required to build each page therefore we made a lot of use of the promise library Q.js to simplify this as well as underscore.js to aggregate the responses into better data structures.
      \item Created a dashboard app to allow configurability of services. Separated presentation from form data definition from request logic to make future extensibility easier. On the presentation layer I added form validation, modal boxes, popups, etc.
      \item Created a way of signing off groups of profile changes as accepted or rejected. The UX chosen was to default to approval while providing a receipt to sign the changes off: this speeds up the approval process while ensuring accountability.
      \item Added features to a leaflet.js map to distinguish driver markers and update the map in real-time.
      \item Created a regulatory area (locale) picker to switch the subdomain in a url.
      \item Created a per-service/endpoint statistics panel which auto-formats data as well as a per-service healthchecks panel and a healthchecks page.
      \item Significant refactoring, integration and bug fixes.
    \end{itemize}
    }}
}

\headedsection
  {\href{http://www.bizzby.com}{BIZZBY}}
  {\textsc{London, United Kingdom}} {%
  \headedsubsection
    {Senior Node.JS Developer}
    {Apr \apo13 -- Jun \apo13}
    {\bodytext{\emph{Bizzby is app your service to help you book a trusted local service in 30 seconds.}
    \begin{itemize}[leftmargin=0cm]
    \item Created APIs to add/remove skills/suggestions from a user and added skill-related popularity functionality to other APIs.
    \item Created service management features to the CMS.
    \end{itemize}
  }}
}

\headedsection
  {\href{http://www.werinteractive.com}{We R Interactive}}
  {\textsc{London, United Kingdom}} {%
  \headedsubsection
    {Lead Node.JS Developer}
    {Oct \apo12 -- Apr \apo13}
    {\bodytext{
        \emph{We R Interactive blends the best of games, film and TV production to create social games that bring global audiences together around sport and music.}
        \begin{itemize}[leftmargin=0cm]
            \item Lead the creation of a service-orientated architecture and helped to recruit and interview developers for my team.
            \item Setup a continuous integration system using Jenkins, and automated deployment of a few of the services to AWS using Puppet.
            \item Created a Node.JS service which would automatically generate a series of questions from a stream of data and a template.
            \item Created a real-time market-outcome resolution service that was able to automatically resolve a series of previously generated questions depending on a stream of real-life data and a simple DSL. This was created using Node.JS, Cassandra, async.js, and underscore.js.
            \item Created a service which could have data pushed to or pulled from it, and that would parse this data into a common format and feed it into a PubSub message queue implemented on top of Redis for deeper processing. An extensive logging system was also implemented to help troubleshoot issues with the third-party supplier of data.
            \item Helped to create a \acr{REST}ful front-end API using Express.js, and also tested the ability to resolve questions in real-time with Socket.io.
            \item Helped to prototype an early version of the mobile app for iOS using Junior.js, Backbone.js, HTML5, CSS3, JavaScript and PhoneGap.
        \end{itemize}
    }}
}

\headedsection
  {\href{http://www.saffrondigital.com}{Saffron Digital}}
  {\textsc{London, United Kingdom}} {%
  \headedsubsection
    {Lead Python Developer}
    {Oct \apo10 -- Oct \apo12}
    {\bodytext{
  \emph{Saffron Digital is the global, market-leading provider of connected device video, DRM, advertising and platform services. Saffron Digital was acquired by HTC in 2011.}
  \begin{itemize}[leftmargin=0cm]
      \item Project leader on the HTC Watch project in which I lead a team of five developers.
      \item Architected the HTC Watch web service (including a 60+ table database). Due to my work with internationalisation and implementation of multiple payment services, this was run globally in close to 20 countries. Later on the project became a baseline for future web services for other clients (LG and Blockbuster).
      \item Significant input in re-engineering development process as we moved from being a startup to a larger company. For example: Git instead of SVN; continuous integration; modern deployment; UltraViolet; etc.
      \item Architecture and development of a distributed encoding orchestration system, coded in Python using Celery with RabbitMQ and later deployed onto AWS. This allows the company to scale up and down any content encoding they were doing depending on demand - and also to support multiple encoders through simple definitions of the AV outputs expected.
      \item Work on a Django key delivery service based on specifications passed to us by UltraViolet.
      \item A client-side application for Samsung Connected TVs and set-top boxes written in object-orientated JavaScript.
  \end{itemize}}
  }
  \headedsubsection
    {PHP Developer}
    {Feb \apo10 -- Oct \apo10}
    {\bodytext{
  \begin{itemize}[leftmargin=0cm]
      \item A RESTful web service for FOX to support a fully localised Family Guy video streaming Android application. A CMS was also created to help enter data, as well as reporting tools to give the business metrics to measure activity.
      \item A web service for Paramount Studio’s the League of Extraordinary Dancers iPhone Application. This was interoperable with the iTunes video store in order to check receipts.
      \item Implemented features on LG Movies on Mobile.  
  \end{itemize}
    }}
}

\headedsection
  {\href{http://gurucareers.com}{Guru Careers}}
  {\textsc{Tonbridge, United Kingdom}} {%
  \headedsubsection
    {Web Developer}
    {Jan \apo10 -- Feb \apo10}
    {\bodytext{Created a Wordpress site that used the \'Buddypress\' plugin to create a social network for the unemployed. Provided jQuery, \acr{CSS}, \acr{PHP} and \acr{XHTML} skills on a redesign.}}
}

\begin{center}
  \emph{Please refer to \href{http://www.linkedin.com/in/heyseb}{my Linkedin profile} for the complete list of work experiences along with recommendations.}
\end{center}


\spacedhrule{-0.2em}{-0.4em}

\roottitle{Education}

\headedsection
  {University of Kent}
  {\textsc{Canterbury, United Kingdom}} {%
  \headedsubsection
    {Bachelor's degree in Computer Science}
    {2005 -- 2009}
    {\bodytext{Courses included Computer Systems and Algorithms, Data Structures and Complexity, Data Mining and Knowledge Discovery, Operating Systems and Architecture, Software Engineering Practice, Database Systems, and Distributed Systems and Networks. While studying here I also coordinated and guided a team of four developers creating a DJ mixing application in Java for my final project.}
  }
}

\headedsection
  {Cranbrook School}
  {\textsc{Cranbrook, United Kingdom}} {%
  \headedsubsection
    {GCSEs \& A-Levels \textnormal{(secondary education)}}
    {2000 -- 2005}
    {\bodytext{}
  }
}

\spacedhrule{0em}{-0.4em}

\roottitle{Interests}

\inlineheadsection
  {Non-exhaustive:}
  {Anthropology, antifragility, complex systems, cognitive science, computational sociology, philosophy, poetry, product design, startups and user experience.}

\end{document}
