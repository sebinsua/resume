% LaTeX source of my resume
% =========================

% Commented for easy reuse... ;)

% See the `README.md` file for more info.

% This file is licensed under the CC-NC-ND Creative Commons license.


% start a document with the here given default font size and paper size
\documentclass[10pt,a4paper]{article}

% include the `tex` instructions that takes care of loading packages and defining commands
% Copyright (c) 2012 Cies Breijs
%
% The MIT License
%
% Permission is hereby granted, free of charge, to any person obtaining a copy
% of this software and associated documentation files (the "Software"), to deal
% in the Software without restriction, including without limitation the rights
% to use, copy, modify, merge, publish, distribute, sublicense, and/or sell
% copies of the Software, and to permit persons to whom the Software is
% furnished to do so, subject to the following conditions:
%
% The above copyright notice and this permission notice shall be included in
% all copies or substantial portions of the Software.
%
% THE SOFTWARE IS PROVIDED "AS IS", WITHOUT WARRANTY OF ANY KIND, EXPRESS OR
% IMPLIED, INCLUDING BUT NOT LIMITED TO THE WARRANTIES OF MERCHANTABILITY,
% FITNESS FOR A PARTICULAR PURPOSE AND NONINFRINGEMENT. IN NO EVENT SHALL THE
% AUTHORS OR COPYRIGHT HOLDERS BE LIABLE FOR ANY CLAIM, DAMAGES OR OTHER
% LIABILITY, WHETHER IN AN ACTION OF CONTRACT, TORT OR OTHERWISE, ARISING FROM,
% OUT OF OR IN CONNECTION WITH THE SOFTWARE OR THE USE OR OTHER DEALINGS IN THE
% SOFTWARE.


% Some commands for making a LaTeX resume
% =======================================

% Commented ;)

% See the README.md file for more info



% \documentclass[10pt,a4paper]{article}  % i do this in the document itself


%%% LOAD AND SETUP PACKAGES

\usepackage[a4paper,margin=0.75in]{geometry}
\usepackage{mdwlist}   % to finetue lists with a inline heading and indented content (see Experiences)
\usepackage{multicol}  % for multiple column text
\usepackage{relsize}   % for \textscale, which I prefer over \sc (small caps), see my \acr command
\usepackage[english]{babel}
\hyphenation{Some-long-word}

\usepackage{enumitem}

\usepackage{hyperref}
%\usepackage[pdftex]{hyperref} % yups, URLs everwhere...
\usepackage{xcolor}  % ... and color them links
\definecolor{dark-blue}{rgb}{0.15,0.15,0.4}
\hypersetup{colorlinks,linkcolor={dark-blue},citecolor={dark-blue},urlcolor={dark-blue}}

\usepackage{ifxetex}
\ifxetex
  \usepackage{fontspec}
  \setmainfont
    [ ExternalLocation = /Users/sebinsua/Library/Fonts/,
      Mapping          = tex-text ,
      Numbers          = OldStyle ,
      Ligatures        = {Common,Contextual} ,
      UprightFont      = *-regular ,
      BoldFont         = *-bold ,
      ItalicFont       = *-italic ,
      BoldItalicFont   = *-bolditalic ]
    {texgyrepagella}
  % Comment out the previous statement and uncomment the following line to use the
  % Linux Libertine font (it has nice lignatures).
  % Make sure to have the `ttf-linux-libertine` package installed on Ubuntu.
  \setmainfont[Mapping=tex-text, Numbers=OldStyle, Ligatures={Common,Contextual}]{Linux Libertine O}
  \usepackage[protrusion]{microtype}  % needs an experimental and impposible to find package for xetex
\else
  \usepackage{tgpagella}  % this case we lack lower case numbers, ligatures and some typographic niceties
  \usepackage[expansion,protrusion]{microtype}
\fi



%%% DOCUMENT WIDE STYLING

\pagestyle{empty}
\setlength{\tabcolsep}{0em}
\xspaceskip7pt  % some more spacing between sentences (use "i.e.\ " or "with SQL\@. " in case of errors)


%%% CUSTOM COMMANDS

% main title (name) with subtitle (date)
\newcommand*\maintitle[2]{\noindent{\LARGE \textbf{#1}}\ \ \ \emph{#2}}

% title for the root sections (experience, education, etc) of the resume
\newcommand*\roottitle[1]{\subsection*{#1}\vspace{-0.3em}\nopagebreak[4]}

% acr command, to quickly mark acronyms for special formatting
\newcommand*\acr[1]{\textscale{.85}{#1}}

% pretty bullet (created from a much smaller centerdot), \sbull contains its spacing
\newcommand*\bull{\raisebox{-0.365em}[-1em][-1em]{\textscale{4}{$\cdot$}}}
\newcommand*\sbull{\ \ \bull \ \ }

% it seems not to work when simply using \parindent...
\newlength{\newparindent}
\addtolength{\newparindent}{\parindent}

% a double \parindent...
\newlength{\doubleparindent}
\addtolength{\doubleparindent}{\parindent}
\addtolength{\doubleparindent}{\parindent}

% indentsection style, used for sections that aren't already in lists
% that need indentation to the level of all text in the document
\newenvironment{indentsection}%
{\begin{list}{}%
  {\setlength{\leftmargin}{\newparindent}\setlength{\parsep}{0pt}\setlength{\parskip}{0pt}\setlength{\itemsep}{0pt}\setlength{\topsep}{0pt}}%
}
{\end{list}}

% headerrow command, used for a new employer
\newcommand{\headedsection}[3]{\nopagebreak[4]\begin{indentsection}\item[]\textscale{1.1}{#1}\hfill#2#3\end{indentsection}\nopagebreak[4]}

% subheaderrow command, used for a new position
\newcommand{\headedsubsection}[3]{\nopagebreak[4]\begin{indentsection}\item[]\textbf{#1}\hfill\emph{#2}#3\end{indentsection}\nopagebreak[4]}

% body text (indented)
\newcommand{\bodytext}[1]{\nopagebreak[4]\begin{indentsection}\item[]#1\end{indentsection}\pagebreak[2]}

% \vspace variaties
\newcommand{\breakvspace}[1]{\pagebreak[2]\vspace{#1}\pagebreak[2]}
\newcommand{\nobreakvspace}[1]{\nopagebreak[4]\vspace{#1}\nopagebreak[4]}

% \spacedhrule a horizontal line with some vertical space before and after it
\newcommand{\spacedhrule}[2]{\breakvspace{#1}\hrule\nobreakvspace{#2}}

% \inlineheadsection command, used for a new employer
\newcommand{\inlineheadsection}[2]{\begin{basedescript}{\setlength{\leftmargin}{\doubleparindent}}\item[\hspace{\newparindent}\textbf{#1}]#2\end{basedescript}\vspace{-1.7em}}

% apo command, for an apostrophe that looks good on old style nums
\newcommand{\apo}{\raisebox{-.18ex}{'}{\hspace{-.1em}}}

% non space that allows line breaks
\newcommand*{\nsp}{\hskip0pt}

%%% MORE SPECIFIC COMMANDS

% CPP command (found it in some corner of the internet and decided to use it)
\newcommand{\CPP}{C\nolinebreak[4]\hspace{-.04em}\raisebox{.20ex}{\footnotesize\bf++} }

% KTurtle command :)
\newcommand{\KTurtle}{\acr{KT}urtle }



% % these are in the document itself:
%
% \begin{document}
% ...the document text...
% \end{document}




\begin{document}  % begin the content of the document
\sloppy  % this to relax whitespacing in favour of straight margins

\maintitle{Seb Insua}{Product-focused Engineer}  % title on top of the document

\nobreakvspace{0.3em}  % add some page break averse vertical spacing

% \noindent prevents paragraph's first lines from indenting
% \mbox is used to obfuscate the email address
% \sbull is a spaced bullet
% \href well..
% \\ breaks the line into a new paragraph
\noindent\href{http://github.com/sebinsua}{github.com/sebinsua}\sbull
\href{http://sebinsua.com}{sebinsua.com}\sbull
\href{https://news.ycombinator.com/user?id=lhnz}{news.yc/sebinsua}\sbull
\href{http://twitter.com/sebinsua}{twitter.com/sebinsua}\sbull

\spacedhrule{0.9em}{-0.4em}  % a horizontal line with some vertical spacing before and after

\roottitle{Summary}  % a root section title

\vspace{-1.3em}  % some vertical spacing
\begin{multicols}{2}  % open a multicolumn environment
\noindent \emph{Systems thinker with a deep interest in the humanities, a passion for product design and an entrepreneurial mindset.}
\\
\\
I was sixteen when my fascination with technology picked up steam. My interest was piqued on discovering a small homebrew community for the handheld gaming device, the GP32. It was inspiring to see a flourishing community of hobbyists creating games for each other and I still maintain that we learn best through play.\newline

I'm product-focused and truly passionate about making an impact in other people's lives. I've always got side projects on-the-go, whether it's the non-profit online music zine that I bootstrapped to 15 regular contributors or the defunct travel startup that I taught myself Objective-C with. I don't always succeed at everything I do, but I give it my best and I try to learn from my mistakes.\newline

My speciality is in combining technical knowledge with lateral thinking. My preferred method of finding insights is to open myself to ideas from multiple domains and form analogies between these. I am fascinated by interdisciplinary thinking and due to this truly appreciate being part of cross-functional teams.\newline

I have a full-stack skillset but recently have worked mainly with JavaScript. I favour techniques from: \href{http://12factor.net}{the twelve-factor app}, functional programming, antifragility with relation to high-availablity systems, behaviour-driven development (\acr{BDD}) and \href{http://infoq.com/presentations/Simple-Made-Easy}{Simple Made Easy}.\newline

\end{multicols}

\inlineheadsection  % special section that has an inline header with a 'hanging' paragraph
  {Leadership, Product \& Operations:}
  {Experience in team leadership (lead teams of 5+ people); multi-faceted approach to product creation: customer development, UX, financial modelling, development; ran a Hackathon with 80 participants, etc.}

\inlineheadsection  % special section that has an inline header with a 'hanging' paragraph
  {Technical specialties:}
  {Full-stack software design and implementation. I love to work on on the back-end creating \acr{REST}ful \acr{API}s using Node.JS or Python; but also know front-end technologies such as JavaScript, \acr{HTML5} and \acr{CSS3}. I follow the mantra \emph{the right tool for the right job} and quickly pick things up.}

\spacedhrule{1.5em}{-0.4em}

\roottitle{Experience}

\headedsection  % sets the header for the section and includes any subsections
  {\href{http://www.momentumlabs.io}{Momentum Labs}}
  {\textsc{London, United Kingdom}} {%

  \headedsubsection  % sets the header for a subsection and contains usually body text
    {MD}
    {Jun \apo13 -- Present}
    {\bodytext{\emph{Hub for my startup ideas.}
    \begin{itemize}[leftmargin=0cm]
      \item My first attempt at a startup was \href{http://getawayapp.co}{an app for spontaneous weekend breaks}. It wasn't the right idea for me to pursue but it allowed me to explore many elements of product creation. I spoke to over one hundred potential users while doing customer development, brainstormed the user experience with my co-founder, and created a native iOS MVP that alongside an attractive pitch successfully got me meetings with insiders. Unfortunately it turned out that travel is both very regulated and an extremely competitive market. After creating a spreadsheet to model the effects of different margins, CAC and conversions we decided that without high-levels of investment it would be difficult to survive to profitability.
      \item Currently I am creating a business networking app. It will allow local communities of helpful people to freely and productively associate with each other in heterogenous work groups by unbundling roles from identities. An early paper prototype has been created as well as a back-end using Neo4j, Redis and Node.JS. An iOS app is being worked on and I'm looking to test my theories with early adopters soon.
    \end{itemize}
  }}

  \headedsubsection  % sets the header for a subsection and contains usually body text
    {Freelancer}
    {Jun \apo13 -- Present}
    {\bodytext{\emph{Freelancing, consulting and open-source activities.}
    \begin{itemize}[leftmargin=0cm]
      \item \href{https://github.com/sebinsua/express-keenio}{An Express.js middleware to allow one-line installation of analytics into Node.JS projects} for the \href{https://www.sequoiacap.com/}{Sequoia Capital} backed Silicon Valley startup \href{http://keen.io}{Keen IO}. This was a project born out of an idea I suggested as a way of reducing the time-to-setup of their analytics service from multiple hours/days to around 15 seconds.
      \item A one-week integration of analytics into \href{http://crowdsurge.com}{CrowdSurge's stores} to help visualise user flows with funnels, segmentation, etc.
      \item A work-in-progress JavaScript object transformation library/philosophy called \href{https://github.com/sebinsua/jstruct}{Jstruct}.
      \item Sporadic articles on user experience, group psychology, technology and philosophy.
    \end{itemize}
  }}
}

\headedsection
  {\href{http://www.red-badger.co.uk}{Red Badger}}
  {\textsc{London, United Kingdom}} {%
  \headedsubsection
    {Contract Node.JS Consultant}
    {Mar \apo14 -- Jul \apo14}
    {\bodytext{\emph{Red Badger is a creative software workshop.}
    \begin{itemize}[leftmargin=0cm]
      \item This was a client-facing role in which I provided training so that key knowledge of a system could be passed from an internal team to an external team.
      \item I worked on features and fixes with technologies including CoffeeScript, Node.JS, RabbitMQ, Elasticsearch and Redis.
    \end{itemize}
    }}
}

\headedsection
  {\href{http://www.hailocab.com}{Hailo}}
  {\textsc{London, United Kingdom}} {%
  \headedsubsection
    {Contract JavaScript Engineer}
    {Sep \apo13 -- Jan \apo14}
    {\bodytext{\emph{Hailo is the evolution of the hail -- a free smartphone app which puts people just two taps away from a licensed taxi, and lets cabbies get more passengers when they want them.}
    \begin{itemize}[leftmargin=0cm]
      \item Created a series of single-page dashboard apps. Conversations with internal customers helped me to come up with ideas on how to simplify business processes. For example a way of signing off groups of driver's profile changes was given a UX that defaulted to approval but forced the user to sign a receipt of the changes: this helps with speed while also ensuring accountability.
      \item Other components were created so that behaviour relating to presentation was well-separated from configuration and data input; or made so that they could be updated in real-time, for example: an animated map of driver locations.
      \item The SOA lacked some orchestration features so building pages sometimes required a lot of HTTP requests. Therefore to simplify this we made a lot of use of the promise library Q.js as well as underscore.js to filter and aggregate responses.
    \end{itemize}
    }}
}

\headedsection
  {\href{http://www.bizzby.com}{BIZZBY}}
  {\textsc{London, United Kingdom}} {%
  \headedsubsection
    {Senior Node.JS Developer}
    {Apr \apo13 -- Jun \apo13}
    {\bodytext{\emph{Bizzby is app your service to help you book a trusted local service in 30 seconds.}
    \begin{itemize}[leftmargin=0cm]
      \item Various APIs and CMS features.
    \end{itemize}
  }}
}

\headedsection
  {\href{http://www.werinteractive.com}{We R Interactive}}
  {\textsc{London, United Kingdom}} {%
  \headedsubsection
    {Lead Node.JS Developer}
    {Oct \apo12 -- Apr \apo13}
    {\bodytext{
        \emph{We R Interactive blends the best of games, film and TV production to create social games that bring global audiences together around sport and music.}
        \begin{itemize}[leftmargin=0cm]
            \item Designed and lead the creation of a back-end system to support a sport game as well as recruiting developers for my team.
            \item The system was decomposed into multiple services and stored data in Cassandra. I created a service which would automatically generate a series of questions from a stream of data and a template, a real-time market-outcome resolution service that would automatically resolve a series of previously generated questions depending on a simple DSL and a stream of real-life data, and a service that could have data pushed to or pulled from it and parse this data into a common format before feeding it into a message queue for deeper processing. I also helped direct my team in prototyping an early \acr{REST}ful front-end API and Phonegap version of the game.
            \item Setup a continuous integration system using Jenkins, and used Puppet to automate deployment of some of the services to AWS. Later on third-party data issues were reported to stakeholders and dealt with through design chances relating to logging and testing the data quality.
        \end{itemize}
    }}
}

\headedsection
  {\href{http://www.saffrondigital.com}{Saffron Digital}}
  {\textsc{London, United Kingdom}} {%
  \headedsubsection
    {Lead Python Developer}
    {Oct \apo10 -- Oct \apo12}
    {\bodytext{
	\emph{Saffron Digital is the global, market-leading provider of connected device video, DRM, advertising and platform services. Saffron Digital was acquired by HTC in 2011.}
	\begin{itemize}[leftmargin=0cm]
	    \item Project leader and architect of the HTC Watch project in which I lead a team of five developers. Due to my early work on internationalisation and implementation of multiple payment services we were able to run this globally in close to 20 countries. Later on the project became a baseline for future web services for other clients.
	    \item Significant input in re-engineering development process as we moved from being a startup to a larger company. For example: an engineer empowering company culture; Git instead of SVN; continuous integration; modern deployment tools; etc.
	    \item Architecture and development of a new platform based on understandings gleaned from previous services. Worked on service to orchestrate and configure encoders in order to run encodes in parallel on AWS.
      \item A client-side application for Samsung Connected TVs and set-top boxes written in object-orientated JavaScript.
	\end{itemize}}
  }
  \headedsubsection
    {PHP Developer}
    {Feb \apo10 -- Oct \apo10}
    {\bodytext{
	\begin{itemize}[leftmargin=0cm]
	    \item A RESTful web service for FOX to support a fully localised Family Guy video streaming Android application. A CMS was also created as well as reporting tools to give the business metrics to measure activity.
	    \item A web service for Paramount Studio’s the League of Extraordinary Dancers iPhone Application. This was interoperable with the iTunes video store in order to check receipts.
	\end{itemize}
    }}
}

\begin{center}
  \emph{Please refer to \href{http://www.linkedin.com/in/heyseb}{my Linkedin profile} for the complete list of work experiences along with recommendations.}
\end{center}


\spacedhrule{-0.2em}{-0.4em}

\roottitle{Education}

\headedsection
  {University of Kent}
  {\textsc{Canterbury, United Kingdom}} {%
  \headedsubsection
    {Bachelor's degree in Computer Science}
    {2005 -- 2009}
    {\bodytext{Courses included Computer Systems and Algorithms, Data Structures and Complexity, Data Mining and Knowledge Discovery, Operating Systems and Architecture, Software Engineering Practice, Database Systems, and Distributed Systems and Networks. While studying here I also coordinated and guided a team of four developers creating a DJ mixing application in Java for my final project.}
  }
}

\headedsection
  {Cranbrook School}
  {\textsc{Cranbrook, United Kingdom}} {%
  \headedsubsection
    {GCSEs \& A-Levels \textnormal{(secondary education)}}
    {2000 -- 2005}
    {\bodytext{}
  }
}

\spacedhrule{0em}{-0.4em}

\roottitle{Interests}

\inlineheadsection
  {Non-exhaustive:}
  {Anthropology, antifragility, economics, sociology, philosophy, poetry, product design, startups and user experience.}

\end{document}
