% LaTeX source of my resume
% =========================

% Commented for easy reuse... ;)

% See the `README.md` file for more info.

% This file is licensed under the CC-NC-ND Creative Commons license.


% start a document with the here given default font size and paper size
\documentclass[10pt,a4paper]{article}

% include the `tex` instructions that takes care of loading packages and defining commands
% Copyright (c) 2012 Cies Breijs
%
% The MIT License
%
% Permission is hereby granted, free of charge, to any person obtaining a copy
% of this software and associated documentation files (the "Software"), to deal
% in the Software without restriction, including without limitation the rights
% to use, copy, modify, merge, publish, distribute, sublicense, and/or sell
% copies of the Software, and to permit persons to whom the Software is
% furnished to do so, subject to the following conditions:
%
% The above copyright notice and this permission notice shall be included in
% all copies or substantial portions of the Software.
%
% THE SOFTWARE IS PROVIDED "AS IS", WITHOUT WARRANTY OF ANY KIND, EXPRESS OR
% IMPLIED, INCLUDING BUT NOT LIMITED TO THE WARRANTIES OF MERCHANTABILITY,
% FITNESS FOR A PARTICULAR PURPOSE AND NONINFRINGEMENT. IN NO EVENT SHALL THE
% AUTHORS OR COPYRIGHT HOLDERS BE LIABLE FOR ANY CLAIM, DAMAGES OR OTHER
% LIABILITY, WHETHER IN AN ACTION OF CONTRACT, TORT OR OTHERWISE, ARISING FROM,
% OUT OF OR IN CONNECTION WITH THE SOFTWARE OR THE USE OR OTHER DEALINGS IN THE
% SOFTWARE.


% Some commands for making a LaTeX resume
% =======================================

% Commented ;)

% See the README.md file for more info



% \documentclass[10pt,a4paper]{article}  % i do this in the document itself


%%% LOAD AND SETUP PACKAGES

\usepackage[a4paper,margin=0.75in]{geometry}
\usepackage{mdwlist}   % to finetue lists with a inline heading and indented content (see Experiences)
\usepackage{multicol}  % for multiple column text
\usepackage{relsize}   % for \textscale, which I prefer over \sc (small caps), see my \acr command
\usepackage[english]{babel}
\hyphenation{Some-long-word}

\usepackage{enumitem}

\usepackage{hyperref}
%\usepackage[pdftex]{hyperref} % yups, URLs everwhere...
\usepackage{xcolor}  % ... and color them links
\definecolor{dark-blue}{rgb}{0.15,0.15,0.4}
\hypersetup{colorlinks,linkcolor={dark-blue},citecolor={dark-blue},urlcolor={dark-blue}}

\usepackage{ifxetex}
\ifxetex
  \usepackage{fontspec}
  \setmainfont
    [ ExternalLocation = /Users/sebinsua/Library/Fonts/,
      Mapping          = tex-text ,
      Numbers          = OldStyle ,
      Ligatures        = {Common,Contextual} ,
      UprightFont      = *-regular ,
      BoldFont         = *-bold ,
      ItalicFont       = *-italic ,
      BoldItalicFont   = *-bolditalic ]
    {texgyrepagella}
  % Comment out the previous statement and uncomment the following line to use the
  % Linux Libertine font (it has nice lignatures).
  % Make sure to have the `ttf-linux-libertine` package installed on Ubuntu.
  \setmainfont[Mapping=tex-text, Numbers=OldStyle, Ligatures={Common,Contextual}]{Linux Libertine O}
  \usepackage[protrusion]{microtype}  % needs an experimental and impposible to find package for xetex
\else
  \usepackage{tgpagella}  % this case we lack lower case numbers, ligatures and some typographic niceties
  \usepackage[expansion,protrusion]{microtype}
\fi



%%% DOCUMENT WIDE STYLING

\pagestyle{empty}
\setlength{\tabcolsep}{0em}
\xspaceskip7pt  % some more spacing between sentences (use "i.e.\ " or "with SQL\@. " in case of errors)


%%% CUSTOM COMMANDS

% main title (name) with subtitle (date)
\newcommand*\maintitle[2]{\noindent{\LARGE \textbf{#1}}\ \ \ \emph{#2}}

% title for the root sections (experience, education, etc) of the resume
\newcommand*\roottitle[1]{\subsection*{#1}\vspace{-0.3em}\nopagebreak[4]}

% acr command, to quickly mark acronyms for special formatting
\newcommand*\acr[1]{\textscale{.85}{#1}}

% pretty bullet (created from a much smaller centerdot), \sbull contains its spacing
\newcommand*\bull{\raisebox{-0.365em}[-1em][-1em]{\textscale{4}{$\cdot$}}}
\newcommand*\sbull{\ \ \bull \ \ }

% it seems not to work when simply using \parindent...
\newlength{\newparindent}
\addtolength{\newparindent}{\parindent}

% a double \parindent...
\newlength{\doubleparindent}
\addtolength{\doubleparindent}{\parindent}
\addtolength{\doubleparindent}{\parindent}

% indentsection style, used for sections that aren't already in lists
% that need indentation to the level of all text in the document
\newenvironment{indentsection}%
{\begin{list}{}%
  {\setlength{\leftmargin}{\newparindent}\setlength{\parsep}{0pt}\setlength{\parskip}{0pt}\setlength{\itemsep}{0pt}\setlength{\topsep}{0pt}}%
}
{\end{list}}

% headerrow command, used for a new employer
\newcommand{\headedsection}[3]{\nopagebreak[4]\begin{indentsection}\item[]\textscale{1.1}{#1}\hfill#2#3\end{indentsection}\nopagebreak[4]}

% subheaderrow command, used for a new position
\newcommand{\headedsubsection}[3]{\nopagebreak[4]\begin{indentsection}\item[]\textbf{#1}\hfill\emph{#2}#3\end{indentsection}\nopagebreak[4]}

% body text (indented)
\newcommand{\bodytext}[1]{\nopagebreak[4]\begin{indentsection}\item[]#1\end{indentsection}\pagebreak[2]}

% \vspace variaties
\newcommand{\breakvspace}[1]{\pagebreak[2]\vspace{#1}\pagebreak[2]}
\newcommand{\nobreakvspace}[1]{\nopagebreak[4]\vspace{#1}\nopagebreak[4]}

% \spacedhrule a horizontal line with some vertical space before and after it
\newcommand{\spacedhrule}[2]{\breakvspace{#1}\hrule\nobreakvspace{#2}}

% \inlineheadsection command, used for a new employer
\newcommand{\inlineheadsection}[2]{\begin{basedescript}{\setlength{\leftmargin}{\doubleparindent}}\item[\hspace{\newparindent}\textbf{#1}]#2\end{basedescript}\vspace{-1.7em}}

% apo command, for an apostrophe that looks good on old style nums
\newcommand{\apo}{\raisebox{-.18ex}{'}{\hspace{-.1em}}}

% non space that allows line breaks
\newcommand*{\nsp}{\hskip0pt}

%%% MORE SPECIFIC COMMANDS

% CPP command (found it in some corner of the internet and decided to use it)
\newcommand{\CPP}{C\nolinebreak[4]\hspace{-.04em}\raisebox{.20ex}{\footnotesize\bf++} }

% KTurtle command :)
\newcommand{\KTurtle}{\acr{KT}urtle }



% % these are in the document itself:
%
% \begin{document}
% ...the document text...
% \end{document}


\begin{document}  % begin the content of the document
\sloppy  % this to relax whitespacing in favour of straight margins

\maintitle{Seb Insua}{Engineering Lead \& Consultant Engineer}  % title on top of the document

\nobreakvspace{0.3em}  % add some page break averse vertical spacing

% \noindent prevents paragraph's first lines from indenting
% \mbox is used to obfuscate the email address
% \sbull is a spaced bullet
% \href well..
% \\ breaks the line into a new paragraph
\noindent
me@sebinsua.com\sbull
07912303758\sbull
\href{http://github.com/sebinsua}{github.com/sebinsua}\sbull
\href{http://sebinsua.com}{sebinsua.com}

\spacedhrule{0.9em}{-0.4em}  % a horizontal line with some vertical spacing before and after

\roottitle{Summary}  % a root section title

\vspace{-1.3em}  % some vertical spacing
\begin{multicols}{2}  % open a multicolumn environment
\\
\\
\noindent I learnt to program at the age of sixteen after discovering a flourishing homebrew community. Realising that self-taught amateurs could create products was inspiring to me. I'm still amazed by what small empowered teams achieve.\newline

\noindent I am comfortable in almost any kind of software engineering position, whether it's developing native apps, tooling, data pipelines, back-end systems or complex real-time applications. I do a lot of commercial work in JavaScript, but in my spare time love developing with Rust due to its unique mix of modern syntax and low-level programming.\newline

\noindent I used to do functional programming but nowadays try to borrow its principles and use these to write beginner-friendly procedural code. Overall, my engineering practices are influenced by the \href{http://12factor.net}{twelve-factor app}, a \href{https://www.tigerteam.dk/2014/micro-services-its-not-only-the-size-that-matters-its-also-how-you-use-them-part-2/}{conservative attitude towards microservices}, and \href{http://infoq.com/presentations/Simple-Made-Easy}{Simple Made Easy}. I follow the mantra \emph{the right tool for the right job} and stay as simple and common as possible.\newline

\noindent I am deeply focused on \emph{finding the right problem}.

\noindent Knowing when not to solve a problem is as important as knowing how to solve it.

\end{multicols}

\spacedhrule{1.5em}{-0.4em}

\roottitle{Career Highlights}

\begin{indentsection}
\item
\begin{itemize}[leftmargin=0cm]

    \item \textbf{Enabled the successful rewrite of a large trading application with a hard deadline.} The deadline was driven by Flash reaching end-of-life. I improved velocity, by modernizing processes and developing tooling, libraries and practices that enabled teams to work in a highly scalable manner. This allowed multiple teams (6-8 teams, 3-5 developers per team) to work independently yet benefit from code re-use, gradually producing 286 separate packages that composed the application.

    \item \textbf{Improved the scalability of development workflows within a large monorepo.} By implementing caching infrastructure we can avoid repeating work.

    \item \textbf{Enabled test workflows that might otherwise take hours to run, to run in under a minute.} By using the container-native workflow engine Argo, I was able to parallelise E2E tests on a Kubernetes cluster.

    \item \textbf{Setup a Data Engineering/Science Practice.} This included hiring a team from scratch, evaluating an original consultancies work, increasing the end-to-end performance of data pipelines by 4x, setting up quality control gates, and arranging training workshops on modern practices.

\end{itemize}
\end{indentsection}

\roottitle{Experience}

\headedsection
  {\href{https://www.jpmorgan.com/country/uk/en/jpmorgan}{JPMorgan Chase \& Co.}}
  {} {%
  \headedsubsection
    {Software Engineer / Architect}
    {Aug \apo19 -- Oct \apo20}
    {\bodytext{
    \begin{itemize}[leftmargin=0cm]
      \item Advised on the technical direction of the group and provided solutions to difficult organisational problems.
      \item Created caching infrastructure to speed up CI. In one case, builds that previously took ~19 minutes took ~3 minutes.
      \item Helped with the upgrade of a large 50+ component library from v3 of Material UI to v4.
      \begin{itemize}[leftmargin=0.39cm]
        \item Introduced TypeScript.
        \item Introduced visual regression testing. 
        \begin{enumerate}[leftmargin=0.39cm]
          \item Auto-generated 100s of tests and then reduced the time these tests took to run from 25 minutes to 10 minutes by developing a codemod that automated rewriting 10,000s of lines of application code to use code-splitting techniques.
          \item Contributed to the open-source libraries `cypress` and `cypress-image-snapshot` and wrote cross-platform code and dockerized Cypress so that it could be used by engineers on Mac, Windows and Linux machines (to avoid image regression failures due due to OS, browser, and font rendering differences).
          \item Used the container-native workflow engine Argo to parallelise the tests on a Kubernetes cluster. This effectively allowed test runs that would otherwise take hours, to run in under a minute.
        \end{enumerate}
      \end{itemize}
    \end{itemize}
    }}
}

\headedsection
  {\href{https://www.shell.com}{Shell}}
  {\textsc{London, United Kingdom}} {%
  \headedsubsection
    {Technical Lead}
    {Dec \apo18 -- Jun \apo19}
    {\bodytext{\emph{Shell is one of the six oil and gas "supermajors" and the fifth-largest company in the world.}
    \begin{itemize}[leftmargin=0cm]
      \item Laid the foundations of a scalable Data Science/Engineering practice.
      \item Starting from zero, hired multiple Data Engineers and Data Scientists into the team (approximately 10 DE/DS). Established a streamlined hiring process for the future.
      \item Reviewed earlier work on the project by a consultancy, and created a plan on how to improve problem areas. Communicated the path forwards to both non-technical stakeholders and globally distributed technical team members. Centred power within a core DE/DS team that would oversee future quality, and setup weekly discussions for thoughtful collaboration and daily stand-ups to stay in sync.
      \item Established a continuous integration process to allow robust improvement of the data engineering and models. This was used to test and document the original code so that it would be understood by future team members.
      \item Prioritised the rapid build of a scalable data platform to enable the new development workflow. This caused the data pipeline to run end-to-end 4x faster, and meant that data engineers could spend more time programming and less time waiting for pipelines to finish.
      \item Helped the team to share their knowledge of modern software engineering practices. Together we created and taught training workshops to teach Data Scientists modern practices for collaborating towards high-quality models (Git, PRs, CI, etc). Outside of the workshops, we used shared repositories to document the onboarding process, how to use Git for collaboration, best practices in EDAs, etc.
      \item Tools: Azure, CircleCI, Docker, Helm, Jupyter, Kubernetes, Python, Spark Cluster, Written Word.
    \end{itemize}
    }}
}

\headedsection
  {\href{https://www.jpmorgan.com/country/uk/en/jpmorgan}{JPMorgan Chase \& Co.}}
  {\textsc{London, United Kingdom}} {%
  \headedsubsection
    {Application Engineer}
    {Jun \apo17 -- Dec \apo18}
    {\bodytext{\emph{JPMorgan Chase \& Co. is an American multinational banking and financial services holding company.}
    \begin{itemize}[leftmargin=0cm]
      \item Facilitated the high-scale software development of a complex realtime trading application built using TypeScript, React and Emotion.
      \item Improved the quality of the PR process by developing a Danger plugin that allowed rapid feedback of real work by automatically deploying sites and linking these to PRs.
      \item Improved modularity of software by demonstrating how to use Rollup and Babel 7 to create best-in-class packages that support CommonJS, ES modules and types.
      \item Improved integration by making an authenticated and declarative API for the loading and code-splitting of sub-applications.
      \item Used working code to teach best practices on higher-order components, render props, performance, hooks, etc.
      \item Evangelised a modern development process leading to multiple teams moving to Lerna monorepos, etc.
    \end{itemize}
    }}
}

\headedsection
  {\href{hhttp://yld.io/}{YLD}}
  {\textsc{London, United Kingdom}} {%
  \headedsubsection
    {Node.js Engineer}
    {Jan \apo17 -- Apr \apo17}
    {\bodytext{\emph{YLD is one of London's fastest growing software engineering consultancies.}
    \begin{itemize}[leftmargin=0cm]
      \item Code reviews. Troubleshooting framework issues. Implementation of 4 Node.js microservices.
      \item Improved the error handling and logging of an in-house microservices framework.
      \item Created shared helpers to ensure a set of data exists during test execution.
      \item Wrote a bit of documentation explaining the process of migration between SQLite and MySQL.
    \end{itemize}
    }}
}

\headedsection
  {\href{http://www.mckinsey.com}{McKinsey \& Company}}
  {\textsc{Western Europe}} {%
  \headedsubsection
    {Full-stack Engineer}
    {Apr \apo16 -- Dec \apo16}
    {\bodytext{\emph{{McKinsey \& Company is one of the "Big Three" management consulting firms.}}
    \begin{itemize}[leftmargin=0cm]
      \item Code reviews. Mentoring. Documentation. Architecture.
      \item D3. React. Redux. Node.js.
      \item Wrote a number of open-source projects (e.g. \href{https://github.com/sebinsua/redux-saga-helpers}{redux-saga-helpers} and \href{https://github.com/sebinsua/react-redux-wizard}{react-redux-wizard}).
    \end{itemize}
    }}
}

\headedsection
  {\href{http://www.economist.com}{The Economist}}
  {\textsc{London, United Kingdom}} {%
  \headedsubsection
    {Full-stack Engineer}
    {Sep \apo15 -- Dec \apo15}
    {\bodytext{\emph{The Economist is an English-language weekly newspaper.}
    \begin{itemize}[leftmargin=0cm]
      \item React. ES6. Node.js.
      \item Numerous components (e.g. \href{https://github.com/economist-components/component-articletemplate}{@economist/component-articletemplate}) used in \href{http://www.theworldin.com}{The World In 2016 project} and potentially other projects.
    \end{itemize}
    }}
}

\headedsection
  {\href{http://gov.uk/government/organisations/home-office}{Home Office}}
  {\textsc{London, United Kingdom}} {%
  \headedsubsection
    {Technology Lead}
    {Jan \apo15 -- Sep \apo15}
    {\bodytext{\emph{The Home Office is a ministerial department of the Government of the United Kingdom, responsible for immigration, security, and law and order.}
    \begin{itemize}[leftmargin=0cm]
      \item Rebuilt a sprawling legacy system into 6 Node.js microservices harnessing infrastructure, services and libraries built for the Passport Exemplar project.
      \item Gave strategic advice on the architecture of a new project, and later presented this plan at 'Show and Tell' to a wide range of people throughout the organisation.
      \item Created some early prototypes using Java 8 + Dropwizard + Swagger.
    \end{itemize}
    }}
}

\pagebreak

\headedsection
  {\href{http://www.red-badger.co.uk}{Red Badger}}
  {\textsc{London, United Kingdom}} {%
  \headedsubsection
    {Node.JS Consultant}
    {Mar \apo14 -- Jul \apo14}
    {\bodytext{\emph{Red Badger is a creative software workshop.}
    \begin{itemize}[leftmargin=0cm]
      \item This was a client-facing role in which I provided training to an external team.
      \item Node.JS, RabbitMQ, Elasticsearch, Redis.
    \end{itemize}
    }}
}

\headedsection
  {\href{http://www.hailocab.com}{Hailo}}
  {\textsc{London, United Kingdom}} {%
  \headedsubsection
    {JavaScript Engineer}
    {Sep \apo13 -- Jan \apo14}
    {\bodytext{\emph{Hailo is the evolution of the hail -- a free smartphone app which puts people just two taps away from a licensed taxi, and lets cabbies get more passengers when they want them.}
    \begin{itemize}[leftmargin=0cm]
      \item Created a series of single-page dashboard apps. Conversations with internal customers helped me to come up with ideas on how to simplify business processes. For example a way of signing off groups of driver's profile changes was given a UX that defaulted to approval but forced the user to sign a receipt of the changes: this helps with speed while also ensuring accountability.
      \item Other components were created so that behaviour relating to presentation was well-separated from configuration and data input; or made so that they could be updated in real-time, for example: an animated map of driver locations.
    \end{itemize}
    }}
}

\headedsection
  {\href{http://www.bizzby.com}{BIZZBY}}
  {\textsc{London, United Kingdom}} {%
  \headedsubsection
    {Senior Node.JS Engineer}
    {Apr \apo13 -- Jun \apo13}
    {\bodytext{\emph{Bizzby is app your service to help you book a trusted local service in 30 seconds.}
  }}
}

\headedsection
  {\href{http://www.werinteractive.com}{We R Interactive}}
  {\textsc{London, United Kingdom}} {%
  \headedsubsection
    {Team Lead}
    {Oct \apo12 -- Apr \apo13}
    {\bodytext{
        \emph{We R Interactive blends the best of games, film and TV production to create social games that bring global audiences together around sport and music.}
        \begin{itemize}[leftmargin=0cm]
            \item Designed and led the creation of a back-end system to support a sport game as well as recruiting developers for my team.
            \item The system was decomposed into multiple services and stored data in Cassandra. I created a service which would automatically generate a series of questions from a stream of data and a template, a real-time market-outcome resolution service that would automatically resolve a series of previously generated questions depending on a simple DSL and a stream of real-life data, and a service that could have data pushed to or pulled from it and parse this data into a common format before feeding it into a message queue for deeper processing.
            \item Setup a continuous integration system using Jenkins, and used Puppet to automate deployment of some of the services to AWS.
        \end{itemize}
    }}
}

\headedsection
  {\href{http://www.saffrondigital.com}{Saffron Digital}}
  {\textsc{London, United Kingdom}} {%
  \headedsubsection
    {Team Lead}
    {Oct \apo10 -- Oct \apo12}
    {\bodytext{
  \emph{Saffron Digital is the global, market-leading provider of connected device video, DRM, advertising and platform services. Saffron Digital was acquired by HTC in 2011.}
  \begin{itemize}[leftmargin=0cm]
      \item Architect on the HTC Watch project in which I later led a team of five developers.
      \item Significant input in re-engineering development process as we moved from being a startup to a larger company.
      \item Architecture and development of a new platform based on understandings gleaned from previous services. Worked on service to orchestrate and configure encoders in order to run encodes in parallel on AWS.
      \item A client-side application for Samsung Connected TVs and set-top boxes written in object-orientated JavaScript.
  \end{itemize}}
  }
  \headedsubsection
    {PHP Developer}
    {Feb \apo10 -- Oct \apo10}
    {\bodytext{
  \begin{itemize}[leftmargin=0cm]
      \item A RESTful web service and CMS or FOX to support a localised Family Guy video streaming Android application.
      \item A web service for Paramount Studio's the League of Extraordinary Dancers iPhone app that integrated with the iTunes video store in order to check receipts.
  \end{itemize}
    }}
}

\begin{center}
  \emph{Please refer to \href{http://www.linkedin.com/in/sebinsua}{my Linkedin profile} for the complete list of work experiences along with recommendations.}
\end{center}


\spacedhrule{-0.2em}{-0.4em}

\roottitle{Education}

\headedsection
  {Coursera}
  {\textsc{Online}} {%
  \headedsubsection
    {Machine Learning}
    {Feb \apo16}
    {\bodytext{\href{https://www.coursera.org/account/accomplishments/verify/E3XLGER56CQ3}{Course Certificate, License E3XLGER56CQ3}}
  }
}

\headedsection
  {University of Kent}
  {\textsc{Canterbury, United Kingdom}} {%
  \headedsubsection
    {Bachelor's degree in Computer Science}
    {2005 -- 2009}
    {\bodytext{}
  }
}

\headedsection
  {Cranbrook School}
  {\textsc{Cranbrook, United Kingdom}} {%
  \headedsubsection
    {GCSEs \& A-Levels \textnormal{(secondary education)}}
    {2000 -- 2005}
    {\bodytext{}
  }
}

\spacedhrule{0em}{-0.4em}

\roottitle{Interests}

\inlineheadsection
  {}
  {Business, Economics, Poetry.}

\end{document}
